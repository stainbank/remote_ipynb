\documentclass[10pt,a4]{article}

\usepackage{fullpage}
\usepackage{setspace}
\usepackage{parskip}
\usepackage{titlesec}
\usepackage{xcolor}
\usepackage{lineno}


% FIGURE PLACEMENT: STRICT
\usepackage[section]{placeins}





\PassOptionsToPackage{hyphens}{url}
\usepackage[colorlinks = true,
            linkcolor = blue,
            urlcolor  = blue,
            citecolor = blue,
            anchorcolor = blue]{hyperref}

\usepackage[natbibapa]{apacite}

%\usepackage{eso-pic}
%\AddToShipoutPictureBG{\AtPageLowerLeft{\includegraphics[scale=0.7]{powered-by-Authorea-watermark.png}}}

\renewenvironment{abstract}
  {{\bfseries\noindent{\large\abstractname}\par\nobreak}}
  {}

\renewenvironment{quote}
  {\begin{tabular}{|p{13cm}}}
  {\end{tabular}}

\titlespacing{\section}{0pt}{*3}{*1}
\titlespacing{\subsection}{0pt}{*2}{*0.5}
\titlespacing{\subsubsection}{0pt}{*1.5}{0pt}


\usepackage{authblk}
\makeatletter
\renewcommand\AB@authnote[1]{\rlap{\textsuperscript{\normalfont#1}}}
\renewcommand\Authsep{,~\,}
\renewcommand\Authands{,~\,and }
\makeatother


\usepackage{graphicx}
\usepackage[space]{grffile}
\usepackage{latexsym}
\usepackage{textcomp}
\usepackage{longtable}
\usepackage{multirow,booktabs}
\usepackage{amsfonts,amsmath,amssymb}
\providecommand\citet{\cite}
\providecommand\citep{\cite}
\providecommand\citealt{\cite}
% You can conditionalize code for latexml or normal latex using this.
\newif\iflatexml\latexmlfalse
\DeclareGraphicsExtensions{.pdf,.PDF,.png,.PNG,.jpg,.JPG,.jpeg,.JPEG}

\usepackage[utf8]{inputenc}
\usepackage[english]{babel}

\usepackage{listings} % for writing source code
\lstset{ % define formatting for source code
  backgroundcolor=\color{white},   % choose the background color
  basicstyle=\footnotesize,        % size of fonts used for the code
  breaklines=true,                 % automatic line breaking only at whitespace
  captionpos=b,                    % sets the caption-position to bottom
  commentstyle=\color{gray},    % comment style
  keywordstyle=\color{blue},       % keyword style
  stringstyle=\color{black},     % string literal style
  frame=lrtb, %
  xleftmargin=\fboxsep, %
  xrightmargin=-\fboxsep, %
  morekeywords={ssh}
}

\begin{document}

\title{Remotely access to Jupyter (IPython) notebooks on the UCL Geography
Linux Cluster}



\author[ ]{Chad Stainbank}%





\vspace{-1em}



  \date{}


\begingroup
\let\center\flushleft
\let\endcenter\endflushleft
\maketitle
\endgroup







The \href{http://jupyter.org/}{Jupyter Notebook}~(previously IPython
Notebook)~provides a feature-rich~interactive environment for learning
and using the \href{https://www.python.org/}{Python Programming
language}. If you're studying for a BSc or MSc at the University College
London Department of Geography then it's likely that you've encountered
these Notebooks in one or more of your modules or in the preparation of
a dissertation. Usually this will be from one of the workstations in
Pearson 110a that forms part of the
\href{http://www.geog.ucl.ac.uk/resources/computer-support/teaching-cluster}{Geography
Linux Cluster}. Since physical access to Pearson 110a is not always
practical, you may wish to have direct access to these files from
another computer. This guide will show you how to \emph{remotely} run
and work with Jupyter Notebooks hosted on the Linux Cluster in a manner
that is fast.

\section{Preamble on operating
systems}

\label{823001}

While the machines in Pearson 110A run a~\emph{Linux} OS, it is likely
that you use either \emph{Mac OS X} or \emph{Microsoft~Windows} on your
own computer. The bulk of this guide is written for somebody using the
terminal on a Linux computer, and Apple fans~should be able to enter all
these same commands on
the~\href{http://www.macworld.co.uk/feature/mac-software/get-more-out-of-os-x-terminal-3608274/}{Mac
Terminal}. The \emph{On Windows~}section adapts these instructions for a
Windows terminal emulator, although it is important to read \emph{On
Linux} first.

However, I strongly urge any Windows users intending on working
extensively with Python and Jupyter Notebooks to consider installing a
Linux distribution. Don't be alarmed: this need not be a drastic switch
as there are two simple ways to set up Linux \emph{alongside} Windows.
The first is to use a
\emph{\href{https://www.howtogeek.com/196060/beginner-geek-how-to-create-and-use-virtual-machines/}{virtual
machine}}, an OS that runs \emph{inside} your current system. This is
very simple to setup (see
\href{http://www.storagecraft.com/blog/the-dead-simple-guide-to-installing-a-linux-virtual-machine-on-windows/}{here})
and I myself use this method to work from UCL's library computers. The
second option, and the one I use for my personal laptop,
is~\emph{\href{https://www.howtogeek.com/187789/dual-booting-explained-how-you-can-have-multiple-operating-systems-on-your-computer/}{dual-booting}.}~Although
this is somewhat involved, there are
\href{https://itsfoss.com/guide-install-linux-mint-16-dual-boot-windows/}{plenty}
\href{https://www.lifewire.com/ultimate-windows-7-ubuntu-linux-dual-boot-guide-2200653}{of}~\href{https://www.howtogeek.com/214571/how-to-dual-boot-linux-on-your-pc/}{resources}
\href{http://www.pcworld.com/article/2955460/operating-systems/dual-booting-linux-with-windows-what-you-need-to-know.html}{available}~to
guide you through the process, and a native Linux install is always
faster than a virtual one. Whichever option you go for, you'll ~need to
choose one from among the many
\emph{\href{http://distrowatch.com/dwres.php?resource=major}{distributions}~}of
Linux.~\href{https://www.ubuntu.com/download}{Ubuntu}~and~\href{https://linuxmint.com/}{Linux
Mint}~are most often suggested for beginners, and for I would recommend
opting for a more lightweight flavour (featuring MATE, Xfce or LXDE as
the desktop environment) for use on a virtual machine or an ageing
laptop.

\section{On Linux/ Mac OS X}

\label{960394}

\subsection{Manual}

\label{450419}

The section provides a very explicit set of instructions to get up and
running with remote Jupyter Notebooks. If you're familiar with the
command-line interface then you could skip to \emph{Automatic}, where
they're condensed to shell functions, but it's probably worth going
through this just once to make sure that everything is working
smoothly.~Here I use the character \textbf{\$} to indicate the line is a
command to be entered, while~\textbf{\#} denotes comments. Anything
enclosed in angle brackets (\textbf{\textless{}\textgreater{}}) must be
replaced, including the brackets themselves, as specified.

\subsubsection{Run Notebook Server}

\label{702963}

The first step is use the program
\href{http://linuxcommand.org/man_pages/ssh1.html}{SSH}~to remotely log
into one of the machines in Pearson 110a and start up a Notebook server.
Because of the way the Linux cluster is set up, this requires a two-step
(or ``multi-hop'') log in. To make the first hop, open a new terminal
and enter the following command, replacing \textless{}USER\textgreater{}
with your UCL username and~\textless{}GATEWAY\textgreater{} with the
name of any one of the gateway servers listed
on~\href{http://www.geog.ucl.ac.uk/resources/computer-support/linux-remote-access}{Geography
Remote Access page}.


\lstset{language=bash}
\lstset{morekeywords={ssh, jupyter, notebook}}

\begin{lstlisting}
$ ssh <USER>@<GATEWAY>.geog.ucl.ac.uk # login to gateway
\end{lstlisting}

You should be prompted for your Geography network password --- note that
this is not necessarily the same password that you use for other UCL
services. After supplying it, enter the command below, replacing
\textless{}MACHINE\textgreater{} with the name of your favourite
teaching workstation in Pearson 110a (see the link above for a list).
Note that, since you're already in the network, you don't need to
provide either your username or the ``.geog.ucl.ac.uk'' address on this
second hop.


\begin{lstlisting}
$ ssh <MACHINE> # login to workstation
\end{lstlisting}

After entering your password again, start running the Notebook server
with the following command. The flag ``--no-browser'' (note the double
dash!) prevents Firefox from opening automatically, while ``--port''
tells the server to listen on a~particular port number. This is 8888 by
default, although you can choose just about any integer above 1026 to
replace \textless{}PORT\textgreater{}.


\begin{lstlisting}
$ jupyter notebook --no-browser --port=<PORT> # run nb server
\end{lstlisting}

Jupyter should now inform you that ``The IPython Notebook is running at:
\textless{}link\textgreater{}''. Don't open the link yet, but look at
the link carefully. Occasionally, the port number (after
"\url{http://localhost}:") will be different from the one you specified.
In this case, simply note down that number and use it as
\textless{}PORT\textgreater{} for subsequent commands.

\subsubsection{Set up SSH tunnel}

\label{813174}

Next, you're going to open a connection to the Notebook server you just
created. Enter the following commands in a~\textbf{new} terminal,
replacing the bracketed values and entering passwords as before. This
time, the ``-L'' flags tell the SSH program to use
\href{https://help.ubuntu.com/community/SSH/OpenSSH/PortForwarding}{local
port forwarding}~so that you can transfer data between your computer and
the server via an
\href{http://blog.trackets.com/2014/05/17/ssh-tunnel-local-and-remote-port-forwarding-explained-with-examples.html}{SSH
tunnel}.


\begin{lstlisting}
$ ssh -L <PORT>:localhost:<PORT> <USER>@<GATEWAY>.geog.ucl.ac.uk # tunnel to gateway
$ ssh -L <PORT>:localhost:<PORT> -N <MACHINE> # tunnel to machine
\end{lstlisting}

You might get a message like:

\begin{quote}
bind: Address already in use

channel\emph{setup}fwd\_listener: cannot listen to port: 8888

Could not request local forwarding"
\end{quote}

This means that another process is using that port and you need to start
the everything again with a different port number. Otherwise, if all has
gone well, the terminal will appear to be hanging~(the ~``-N'' flag
prevents any more user input) and the SSH tunnel should be open. To use
the Notebook, open a browser on your computer and paste in the link
provided in the first terminal (e.g.~\url{http://localhost:8888}).
Congratulations, you should now be looking at the familiar Jupyter
Notebook file browser.


\subsection{Automatic method}

\label{362547}

It will very quickly become tedious to type out these commands perfectly
every time you want to run a Notebook remotely. The two
(\href{http://cs.lmu.edu/~ray/notes/bash/}{bash})~\href{https://www.shellscript.sh/functions.html}{functions}~below
can be used to condense these lines to a few short commands.


\begin{lstlisting}
# Replace these defaults
GUSER=ucfacms
GATEWAY=arch
MACHINE=freetown
GPORT=9000
# Log in UCL Geography Linux cluster machine
geog () {
    ssh -t -Y ${3:-$GUSER}@${2:-$GATEWAY}.geog.ucl.ac.uk ssh -Y ${1:-$MACHINE}
}
# Tunnel into UCL Geography Linux cluster machine
geogport () {
    PORT=${1:-$GPORT}
    ssh -L $PORT:localhost:$PORT ${4:-$GUSER}@${3:-$GATEWAY}.geog.ucl.ac.uk ssh -L $PORT:localhost:$PORT -N ${2:-$MACHINE}
}

\end{lstlisting}

To ``install'' these two functions, open
your~\href{http://superuser.com/questions/49289/what-is-the-bashrc-file}{.bashrc}~file
(located in the user root directory) with a text editor and paste in the
entire code block, making the appropriate replacements to the default
values before saving. If this file doesn't exist
(\href{http://apple.stackexchange.com/a/119714}{as is the case on Mac OS
X}) then simply create it, then add the following line to the file
.bash\_profile~(also in the user root directory):


\begin{lstlisting}
# Source bashrc
if [ -f ~/.bashrc ]; then . ~/.bashrc; fi
\end{lstlisting}

\subsubsection{Generate SSH keys}

\label{951039}

However, the \emph{geogport} function will not work if you have to type
in your password when logging in from one geography machine to another.
One way to authenticate yourself for login without a password is to set
up a pair of \href{https://wiki.archlinux.org/index.php/SSH_keys}{SSH
keys}. This is very easy to do; the following instructions are adapted
from
\href{https://www.digitalocean.com/community/tutorials/how-to-set-up-ssh-keys--2}{this
guide}. First, log in to a machine on the geography Linux cluster
(either physically or via SSH as described above) and enter the
command:~


\begin{lstlisting}
ssh-keygen -t rsa # generate ssh keys
\end{lstlisting}

Let it use the default file and use no passphrase (i.e. press enter
twice)

\section{On Windows}

\label{909619}

Working with Jupyter Notebooks requires some set-up before adapting the
instructions in~\emph{On Linux}, although you must \textbf{read that
section first}.

\subsection{Windows Subsystem for
Linux}

\label{811558}

If you have Windows 10 then you'll be able to take advantage of a new
Microsoft development called \emph{Windows Subsystem for Linux} (WSL),
otherwise referred to as \emph{Bash on Ubuntu on Linux}, which provides
a Linux command-line interface on Windows. To use WSL to run Jupyter
Notebooks, simply
\href{https://msdn.microsoft.com/en-gb/commandline/wsl/install_guide}{install
the feature} then follow the instructions above for Linux.

\subsection{PuTTY}

\label{903289}

If you don't have access to WSL then you'll have to use the popular
terminal emulator and SSH client \emph{PuTTY~}to access your Notebooks.
First install the program from
\href{http://www.chiark.greenend.org.uk/~sgtatham/putty/latest.html}{its
website}, then run a PuTTY instance to load its GUI configuration
screen. Navigate to the ``Session'' and enter

\textless{}USER\textgreater{}@\textless{}GATEWAY\textgreater{}.geog.ucl.ac.uk
in the ``Host Name'' box, making the appropriate replacements as
described in \emph{On Linux/Mac OS X.~}Follow
\href{http://www.geo.mtu.edu/geoschem/docs/putty_install.html}{this
guide} if you want to be able to load X11 windows (i.e. graphical
programs such as~\emph{gedit}~or~\emph{Firefox}), although this is not
necessary for using Notebooks. To store these gateway login settings,
enter a name under \emph{Saved Sessions} and click \emph{Save}.


\begin{figure}[h!]
\begin{center}
\includegraphics[width=0.70\columnwidth]{{{figures/5. Save putty settings/5. Save putty settings}}}
\caption{{Enter host name and save gateway settings.
\label{673841}%
}}
\end{center}
\end{figure}

To set up the SSH tunnel, load your gateway settings and then navigate
to the port forwarding options (\emph{Connection} \textgreater{}
\emph{SSH} \textgreater{} \emph{Auth} \textgreater{} \emph{Tunnels}).
Under \emph{Source Port}~supply~\textless{}PORT\textgreater{} and enter
127.0.0.1:\textless{}PORT\textgreater{} as the \emph{Destination},
replacing \textless{}PORT\textgreater{}~with the port number of your
choice (as described above), then\emph{~c}lick~\emph{Add} to commit this
information to the \emph{Forwarded ports} list.


\begin{figure}[h!]
\begin{center}
\includegraphics[width=0.70\columnwidth]{{{figures/7. Confirm port forwarding/7. Confirm port forwarding}}}
\caption{{Set up port forwarding.
\label{994371}%
}}
\end{center}
\end{figure}

Finally, save these settings by navigating back to~\emph{Sessions},
entering a \textbf{new} name under \emph{Saved Sessions} and
clicking~\emph{Save},~then close PuTTY.


\begin{figure}[h!]
\begin{center}
\includegraphics[width=0.70\columnwidth]{{{figures/8. Save putty port settings/8. Save putty port settings}}}
\caption{{Save gateway tunneling session.
\label{665311}%
}}
\end{center}
\end{figure}

To run and work with Jupyter Notebook on the Geography Linux Cluster,
open two instances of PuTTY. Double click~the name of the first saved
session (``arch'' in the screenshots) in one window and then open the
second session (i.e. ``arch-port'') in the other. Considering that these
windows respectively serve to replace the commands~commented ``login to
gateway'' and ``tunnel to gateway'' in the section~\emph{On Linux/ Mac
OS X,~}simply complete the remaining instructions to run and connect to
the Notebook server.

\end{document}

