\documentclass[a4paper]{article}

\usepackage{listings}
\lstset{language=bash, frame=single}
\usepackage[pdftex]{hyperref} % must go last!

% Title
\title{Remotely work with Jupyter(IPython) notebooks on the UCL Geography Linux cluster}
\author{Chad Stainbank}

\begin{document}
\maketitle
\section{Introduction}
\subsection{What is this?}

The \href{http://jupyter.org/}{Jupyter Notebook}~(previously IPython Notebook)~provides a feature-rich~interactive environment for learning and using the \href{https://www.python.org/}{Python Programming language}.
If you're studying for a BSc or MSc at the University College London Department of Geography then it's likely that you've encountered these Notebooks in one or more of your modules or in the preparation of a dissertation.
Usually this will be from one of the workstations in Pearson 110a that forms part of the \href{http://www.geog.ucl.ac.uk/resources/computer-support/teaching-cluster}{Geography Linux Cluster}.
Since physical access to Pearson 110a is not always practical, you may wish to have direct access to these files from another computer.
This guide will show you how to \emph{remotely} run and work with Jupyter Notebooks hosted on the Linux Cluster in a manner that is fast.

\subsection{Preamble on operating systems}

While the machines in Pearson 110A run a~\emph{Linux} OS, it is likely that you use either \emph{Mac OS X} or \emph{Microsoft~Windows} on your own computer. The bulk of this guide is written for somebody using the terminal on a Linux computer, and Apple fans~should be able to enter all these same commands on the~\href{http://www.macworld.co.uk/feature/mac-software/get-more-out-of-os-x-terminal-3608274/}{Mac Terminal}. The \emph{On Windows~}section adapts these instructions for a Windows terminal emulator, although it is important to read \emph{On Linux} first.

However, I strongly urge any Windows users intending on working extensively with Python and Jupyter Notebooks to consider installing a Linux distribution.
Don't be alarmed: this need not be a drastic switch as there are two simple ways to set up Linux \emph{alongside} Windows.
The first is to use a \emph{\href{https://www.howtogeek.com/196060/beginner-geek-how-to-create-and-use-virtual-machines/}{virtual machine}}, an OS that runs \emph{inside} your current system.
This is very simple to setup (see \href{http://www.storagecraft.com/blog/the-dead-simple-guide-to-installing-a-linux-virtual-machine-on-windows/}{here}) and I myself use this method to work from UCL's library computers.
The second option, and the one I use for my personal computer, is~\emph{\href{https://www.howtogeek.com/187789/dual-booting-explained-how-you-can-have-multiple-operating-systems-on-your-computer/}{dual-booting}.}~Although this is somewhat involved, a native Linux install is always faster than a virtual one and there are plenty of guides available online to guide you through the process (e.g. \href{https://itsfoss.com/guide-install-linux-mint-16-dual-boot-windows/}{1} \href{https://www.lifewire.com/ultimate-windows-7-ubuntu-linux-dual-boot-guide-2200653}{2}~\href{https://www.howtogeek.com/214571/how-to-dual-boot-linux-on-your-pc/}{3} \href{http://www.pcworld.com/article/2955460/operating-systems/dual-booting-linux-with-windows-what-you-need-to-know.html}{4}). 
Whichever option you go for, you'll ~need to choose one from among the many \emph{\href{http://distrowatch.com/dwres.php?resource=major}{distributions}~}of Linux.
\href{https://www.ubuntu.com/download}{Ubuntu}~and~\href{https://linuxmint.com/}{Linux Mint}~are most often suggested for beginners, and I would recommend opting for a more lightweight flavour (featuring MATE, Xfce or LXDE as the desktop environment) for use on a virtual machine or an ageing laptop.

\section{On Linux/ Mac OS X}
\subsection{Manual}
\label{sec:manual}

The section provides a very explicit set of instructions to get up and running with remote Jupyter Notebooks.
The listings boxes display commands to be entered into the terminal. \textbf{\textless{}\textgreater{}} enclose values which must be replaced as specified, while comments --- any text after \textbf{\#} --- do not need to be entered.
Since these commands may not always render properly, be careful to distinguish one dash ``-'' from two ``-{}-'' --- it should never be a long dash ``--''!
If you're familiar with the command-line interface then you could skip to section \ref{sec:auto}, where they're condensed to shell functions, but it's probably worth going through this just once to make sure that everything is working smoothly.

\subsubsection{Run Notebook Server}
\label{sec:runnb}

The first step is use the program \href{http://linuxcommand.org/man_pages/ssh1.html}{Secure Shell (SSH)}~to remotely log into one of the machines in Pearson 110a and start up a Notebook server.
Because of the way the Linux cluster is set up, this requires a two-step (or ``multi-hop'') log in.
To make the first hop, open a new terminal and enter the command in listing \ref{lst:logingw}, replacing \textless{}USER\textgreater{} with your UCL username and~\textless{}GATEWAY\textgreater{} with the name of any one of the gateway servers listed on~\href{http://www.geog.ucl.ac.uk/resources/computer-support/linux-remote-access}{Geography Remote Access page}.

\begin{lstlisting}[caption={Login to gateway}, label={lst:logingws}]
ssh <USER>@<GATEWAY>.geog.ucl.ac.uk
\end{lstlisting}

You should be prompted for your Geography network password --- note that this is not necessarily the same password that you use for other UCL services.
After supplying it, enter the command in \ref{lst:loginws}, replacing \textless{}MACHINE\textgreater{} with the name of your favourite teaching workstation in Pearson 110a \footnote{Note that, since you're already in the network, you can omit your username and ``.geog.ucl.ac.uk'' on the second hop.}(see href{http://www.geog.ucl.ac.uk/resources/computer-support/linux-remote-access}{here} for a list). 

\begin{lstlisting}[caption={Login to workstation}, label={lst:loginws}]
ssh <MACHINE>
\end{lstlisting}

After entering your password again, start running the Notebook server with command \ref{lst:runnb}.
The flag ``--no-browser'' prevents Firefox from opening automatically, while ``--port'' tells the server to listen on a particular port number.
This is 8888 by default, although you can choose just about any integer above 1026 to replace \textless{}PORT\textgreater{}.

\begin{lstlisting}[caption={Run Notebook server}, label={lst:runnb}]
jupyter notebook --no-browser --port=<PORT>
\end{lstlisting}

Jupyter should now inform you that ``The IPython Notebook is running at: \textless{}link\textgreater{}''.
Without opening the link, check that the port number (after ``http://localhost:'') is the same the one you specified.
If it has changed, simply note down the new number and use it as \textless{}PORT\textgreater{} for all subsequent commands.

\subsubsection{Set up SSH tunnel}

Next, you're going to open a connection to the Notebook server you just created.
In a~\textbf{new} terminal, enter the commands in listings \ref{lst:tunnel1} and \ref{lst:tunnel2}. As before, replace the bracketed values and supply your password when prompted.
This time, the ``-L'' flags tell SSH to use \href{https://help.ubuntu.com/community/SSH/OpenSSH/PortForwarding}{local port forwarding}~so that you can transfer data between your computer and the server via an \href{http://blog.trackets.com/2014/05/17/ssh-tunnel-local-and-remote-port-forwarding-explained-with-examples.html}{SSH tunnel}.

\begin{lstlisting}[caption={Tunnel to gateway server}, label={lst:tunnel1}]
ssh -L <PORT>:localhost:<PORT> -N <MACHINE>
\end{lstlisting}

\begin{lstlisting}[caption={Tunnel to machine}, label={lst:tunnel2}]
ssh -L <PORT>:localhost:<PORT> -N <MACHINE>
\end{lstlisting}

If all has gone well, the terminal will appear to be hanging \footnote{The optional ``-N'' flag prevents any more user input} and the SSH tunnel should be open.
To use the Notebook, copy the link produced in section \ref{sec:runnb} (e.g.~\url{http://localhost:8888}) and paste it into the URL bar of your favourite browser. 
Congratulations, you should now be looking at the familiar Jupyter Notebook file browser. 

\subsubsection{Issues}

Occasionally, when trying to set up the SSH tunnel, you will get a message like:
\begin{quote}
bind: Address already in use

channel\emph{setup}fwd\_listener: cannot listen to port: \textless{}PORT\textgreater{}

Could not request local forwarding
\end{quote}

This means that the tunnel could not be established as another process is using that particular port.
If this occurs, start everything over again with a different port number.

\subsection{Automatic}
\label{sec:auto}

\subsubsection{Generate SSH keys}
\label{sec:sshkeys}
Following all the steps in \ref{sec:manual} require you to enter your password a total of four times, once for each SSH command.
\href{https://wiki.archlinux.org/index.php/SSH_keys}{SSH keys} provide an alternative form of authentication, using a private and a public key to allow passwordless login over SSH.
Two seperate pairs of keys must be set-up: one for the initial hop from your computer to the gateway node, and the other for the onward hop to the workstation.

To set up the first pair of keys, open a terminal on your computer and enter the command:
\begin{lstlisting}[caption={Generate SSH keys}, label={lst:genkeys}]
ssh-keygen -t rsa
\end{lstlisting}
Let the program use the default file with no passphrase by presssing enter twice, then copy the public key to the geography file network using command \ref{lst:cpkeys} and following the instructions returned by the program.

\begin{lstlisting}[caption={Copy public key to geography file system}, label={lst:cpkeys}]
ssh-copy-id <USER>@<GATEWAY>.geog.ucl.ac.uk
\end{lstlisting}

To generate a pair of keys for the second hop, simply login to the any machine on the linux cluster and repeat the process.
\footnote{You do not need to specify \textless{}USER\textgreater{}, nor finish the address with ``.geog.ucl.ac.uk'', as you are already within the geography network.}

\subsection{Geography login/tunnel functions}
\label{sec:gfuncs}

It will very quickly become tedious to perfectly type out the commands in section \href{sec:manual} every time you want to run a Notebook remotely.
The two (\href{http://cs.lmu.edu/~ray/notes/bash/}{bash})~\href{https://www.shellscript.sh/functions.html}{functions}~in listing \ref{lst:gfuncs} can be used to condense these lines to a few short commands.

\lstinputlisting[float, floatplacement=hpb, caption={Functions to login/tunnel to geography cluster}, label={lst:gfuncs}, firstline=2, lastline=21, breaklines=true]{geog_functions.sh}

Be aware that listing \ref{lst:gfuncs} is for demonstration only; it does not display properly in a pdf document and so will not work if used as-is.
Instead, raw text file containing these functions can be found at:
{}
\noindent\url{https://github.com/stainbank/remote_ipynb/blob/master/geog_functions.sh}
{}
The simplest way to ``install'' these two functions is to paste the entire code block --- replacing the defaults as appropriate --- into a file in your home directory called \href{http://superuser.com/questions/49289/what-is-the-bashrc-file}{.bashrc}~file.
If this file doesn't already exist, \href{http://apple.stackexchange.com/a/119714}{as is the case on Mac OS X}, simply create it and then add the following line to the file .bash\_profile~(also in the user root directory):

\begin{lstlisting}[caption={Source .bashrc on startup}, label={lst:srcbashrc}]
if [ -f ~/.bashrc ]; then . ~/.bashrc; fi
\end{lstlisting}

The command \emph{geog} replaces the commands in listings \ref{lst:logingw} and \ref{lst:loginws}, while \emph{geogport} replaces \ref{lst:tunnel1} and \ref{lst:tunnel2}.
You can therefore work with a remote notebook by simply entering:
\begin{lstlisting}[caption={Set up and tunnel to remote Notebook server}, label={lst:usegfuncs}]
# In one terminal
geog
jupyter notebook --no-browser --port=<PORT>
# In another terminal
geogport
\end{lstlisting}

These functions also take positional arguments, in the order \emph{geog} \textless{}MACHINE\textgreater{} \textless{}GATEWAY\textgreater{} \textless{}USER\textgreater{} and \emph{geogport} \textless{}PORT\textgreater{} \textless{}MACHINE\textgreater{} \textless{}GATEWAY\textgreater{} \textless{}USER\textgreater{}.
This allows you to change any variables from their default values \footnote{You must supply all \emph{preceeding} arguments; prefix the an argument name with \textbf{\$} to use the default value}:

\begin{lstlisting}[caption={Examples of arguments to geography login/tunnel functions}, label={lst:gfuncsargseg}]
geog ulanbator # change machine
geog $MACHINE triangle ucfaxyz # change user and gateway
geogport 8889 # change port
geogport $GPORT lima # change machine
\end{lstlisting}

\end{document}
